\documentclass{article}

\usepackage{amsfonts}
\usepackage{amsthm}

\theoremstyle{definition}
\newtheorem{definition}{Definition}

\begin{document}

\title{Smooth Hashlife: An improvement of the Hashlife algorithm to detect
non-power-of-two displacements}
\author{Itai Bar-Natan}
\maketitle

\tableofcontents

\section{Introduction}

In this article I describe an algorithm and implemention of that algorithm to
simulate cellular automata

\section{Cellular automata}

A cellular automaton is a set of rules describing a grid of cells, where each
cell is in one of a finite set of states, and how the state each cell changes
over time as a function of the states of the nearby cells. We can describe this
more precisely as follows: A cellular automaton has a set $S$ (typically finite)
of states. A \emph {cell configuration} for this set of states is a function
$\alpha : \mathbb {Z}^2 \to S$ which describes the state at each point on the
two-dimensional square lattice. Given a cell configuration $\alpha$, as well as
integers $m_0 \leq m_1$ and $n_0 \leq n_1$, we can consider a \emph {rectangle}
in $\alpha$ to be the restriction of $\alpha$ to $([m_0, m_1] \times [n_0, n_1])
\cap \mathbb {Z}^2$, thought of as an $(m_1-m_0+1) \times (n_1-n_0+1)$ matrix
with entries in $S$. We will abuse notation slightly and write this matrix as
$\alpha |_{[m_0, m_1] \times [n_0, n_1]}$; in particular, with this notation it
is possible that $\alpha _{[m, m+k] \times [n, n+\ell]} = \beta _{[m', m'+k]
\times [n', n'+\ell]}$ even when $(m, n) \neq (m', n')$, unlike with ordinary
restriction. Finally, a \emph {cellular automaton} consists of the set of states
$S$ as well as a \emph {transition function} $t : M _{3, 3} (S) \to S$.  Given a
cellular automaton $(S, t)$ and a cellular configuration $\alpha$, we can define
the \emph {evolution of $\alpha$} as sequence of cell configurations $(\alpha_n)
_{n \geq 0}$ defined inductively by $\alpha_0 = \alpha$ and

$$ \alpha _{n+1} (i, j) = t (\alpha_n |_{[i-1, i+1] \times [j-1, j+1]}) $$

The definition above can be generalized in many ways. For instance, we can
consider cellular automata in dimensions other than two, have the state of a
cell be affected by a different set of neighbors rather than the Moore
neighborhood, or use a lattice other than the square lattice. One modification
that will seem abstruse but turn out useful is to have a lattice for \emph
{spacetime} that does not decompose as a spatial lattice and an integer time.

Specifically, can consider the lattice $L = \{(x, y, t) \in \mathbb {Z}^3} | x +
y + t \in 2 \mathbb {Z}\}$. In this lattice, we can consider cellular automata
where the state in cell $(x, y, t)$ depends on the states in the four cells $(x
\pm 1, y \pm 1, t-1)$.

\section{Applications of Cellular Automata}

They're pretty.

\section{Description of Hashlife}

\section{Focal Points}

TODO Motivation

\begin{definition}
Fix two increasing sequences $(a_n), (b_n)$ of positive integers. Given a cell
configuration, we define the \emph{$n$-focal points} by induction on $n$:

TODO Define/clarify cell configuration, square, center of square, hash

\begin{itemize}
\item Any point on the grid is a $0$-focal point.
\item An $n$-focal point is an $(n-1)$-focal point $(x, y)$, with the property
that the $a_n \times a_n$ rectangle centered at $(x, y)$ has a larger hash value
than any $a_n \times a_n$ rectangle centered at $(x', y')$ where $(x', y')$ is
$(n-1)$-focal and $|x - x'|, |y - y'| \leq b_n$.
\end{itemize}
\end{definition}

The $n$-focal points have the following properties:

\begin{itemize}
\item Translation invariance: Translating a cell configuration by $(a, b)$
translates all the $n$-focal points by $(a, b)$.

\item Locality: It is possible to tell whether a point $(x, y)$ is $n$-focal
given only the knowledge of the states of the cells in the diameter $a_n + 2
b_n$ square around $(x, y)$.

\item Sparsity: Any $b_n \times b_n$ square contains at most one $n$-focal cell.

\item Efficient computation: Assuming $(a_n)$ $(b_n)$ grow exponentially with
the same constant, the time it takes to calculate all the $i$-focal points in an
$n \times n$ square is $O (i n^2)$. Typically we're interested in $i \sim \log
n$ leading to an asymptotic cost of $O (n^2 \log n)$.
\end{itemize}

\end{document}
