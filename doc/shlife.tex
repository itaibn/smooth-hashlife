\documentclass{article}

\usepackage{amsthm}

\theoremstyle{definition}
\newtheorem{definition}{Definition}

\begin{document}

\title{Smooth Hashlife: An improvement of the Hashlife algorithm to detect
non-power-of-two displacements}
\author{Itai Bar-Natan}
\maketitle

\section{Introduction}

In this article I describe an algorithm and implemention of that algorithm to
simulate cellular automata

\section{Cellular automata}

A cellular automaton is a set of rules describing a grid of cells, where each
cell is in one of a finite set of states, and how the state each cell changes
over time as a function of the states of the nearby cells. We can describe this
more precisely as follows: 

\section{Description of Hashlife}

\section{Focal Points}

TODO Motivation

\begin{definition}
Fix two increasing sequences $(a_n), (b_n)$ of positive integers. Given a cell
configuration, we define the \emph{$n$-focal points} by induction on $n$:

TODO Define/clarify cell configuration, square, center of square, hash

\begin{itemize}
\item Any point on the grid is a $0$-focal point.
\item An $n$-focal point is an $(n-1)$-focal point $(x, y)$, with the property
that the $a_n \times a_n$ rectangle centered at $(x, y)$ has a larger hash value
than any $a_n \times a_n$ rectangle centered at $(x', y')$ where $(x', y')$ is
$(n-1)$-focal and $|x - x'|, |y - y'| \leq b_n$.
\end{itemize}
\end{definition}

The $n$-focal points have the following properties:

\begin{itemize}
\item Translation invariance: Translating a cell configuration by $(a, b)$
translates all the $n$-focal points by $(a, b)$.

\item Locality: It is possible to tell whether a point $(x, y)$ is $n$-focal
given only the knowledge of the states of the cells in the diameter $a_n + 2
b_n$ square around $(x, y)$.

\item Sparsity: Any $b_n \times b_n$ square contains at most one $n$-focal cell.

\item Efficient computation: Assuming $(a_n)$ $(b_n)$ grow exponentially with
the same constant, the time it takes to calculate all the $i$-focal points in an
$n \times n$ square is $O (i n^2)$. Typically we're interested in $i \sim \log
n$ leading to an asymptotic cost of $O (n^2 \log n)$.
\end{itemize}

\end{document}
